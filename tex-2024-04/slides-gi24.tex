\documentclass[dvipsnames,aspectratio=169,pdftex]{beamer}
\usepackage{agda}
\usepackage{stmaryrd}
\usepackage{xcolor}
\usepackage{txfonts}
\usepackage[T1]{fontenc}
\usepackage{microtype}
\DisableLigatures[-]{encoding=T1}
\usepackage{tikz}
\usetikzlibrary{cd}
\usepackage{agda-generated}


\input{unicodeletters}
\newcommand\Aamp{\AgdaFunction{\ensuremath{\&}}}
\newcommand\Att{\AgdaInductiveConstructor{tt}}
\newcommand\Ado{\AgdaKeyword{do}}
\newcommand\AZ{\AgdaDatatype{ℤ}}
\newcommand\Asuc{\AgdaInductiveConstructor{suc}}
\newcommand\Azero{\AgdaInductiveConstructor{zero}}
\newcommand\Anat{\AgdaInductiveConstructor{nat}}
\newcommand\Aint{\AgdaInductiveConstructor{int}}
\newcommand\Abool{\AgdaInductiveConstructor{bool}}
\newcommand\Atend{\AgdaInductiveConstructor{end}}
\newcommand\Atsend[2]{\AgdaInductiveConstructor{‼{\textcolor{black}{\ensuremath{#1}}}∙{\textcolor{black}{\ensuremath{#2}}}}}
\newcommand\Atrecv[2]{\AgdaInductiveConstructor{⁇{\textcolor{black}{\ensuremath{#1}}}∙{\textcolor{black}{\ensuremath{#2}}}}}
\newcommand\Atcfsend[1]{\AgdaInductiveConstructor{‼{\textcolor{black}{\ensuremath{#1}}}}}
\newcommand\Atcfrecv[1]{\AgdaInductiveConstructor{⁇{\textcolor{black}{\ensuremath{#1}}}}}
\newcommand\Atcfcomp[2]{\AgdaInductiveConstructor{{\textcolor{black}{\ensuremath{#1}}}⨟{\textcolor{black}{\ensuremath{#2}}}}}
\newcommand\Atcfskip{\AgdaInductiveConstructor{skip}}
\newcommand\ACSKIP{\AgdaInductiveConstructor{SKIP}}
\newcommand\ACEND{\AgdaInductiveConstructor{END}}
\newcommand\ACCLOSE{\AgdaInductiveConstructor{CLOSE}}
\newcommand\ACfork{\AgdaInductiveConstructor{fork}}
\newcommand\ACconnect{\AgdaInductiveConstructor{connect}}
\newcommand\ACterminate{\AgdaInductiveConstructor{terminate}}
\newcommand\ACdelegateIN{\AgdaInductiveConstructor{delegateIN}}
\newcommand\ACdelegateOUT{\AgdaInductiveConstructor{delegateOUT}}
\newcommand\ACtransmit{\AgdaInductiveConstructor{transmit}}
\newcommand\ACbranch{\AgdaInductiveConstructor{branch}}
\newcommand\ACclose{\AgdaInductiveConstructor{close}}
\newcommand\ACSEND{\AgdaInductiveConstructor{SEND}}
\newcommand\ACRECV{\AgdaInductiveConstructor{RECV}}
\newcommand\ACSELECT{\AgdaInductiveConstructor{SELECT}}
\newcommand\ACCHOICE{\AgdaInductiveConstructor{CHOICE}}
\newcommand\AFin{\AgdaDatatype{Fin}}
\newcommand\AXCommand{\AgdaDatatype{XCmd}}
\newcommand\ACommand{\AgdaDatatype{Cmd}}
\newcommand\ACommandStack{\AgdaDatatype{CmdStack}}
\newcommand\ASession{\AgdaDatatype{Session}}
\newcommand\ASplit{\AgdaDatatype{Split}}
\newcommand\AMSession{\AgdaDatatype{MSession}}
\newcommand\ASet{\AgdaDatatype{Set}}
\newcommand\ASetOne{\AgdaDatatype{Set$_1$}}
\newcommand\ASeto{\AgdaDatatype{Setω}}
\newcommand\Abinaryp{\AgdaFunction{binaryp}}
\newcommand\Aunaryp{\AgdaFunction{unaryp}}
\newcommand\ACheck{\AgdaFunction{Check}}
\newcommand\ACausality{\AgdaFunction{Causality}}
\newcommand\ACheckDual{\AgdaFunction{CheckDual0}}
\newcommand\Aexecutor{\AgdaFunction{exec}}
\newcommand\Aexec{\AgdaFunction{exec}}
\newcommand\AIO{\AgdaFunction{IO}}
\newcommand\Amu{\AgdaInductiveConstructor{\ensuremath{\mu}}}
\newcommand\AMU{\AgdaInductiveConstructor{LOOP}}
\newcommand\AUNROLL{\AgdaInductiveConstructor{UNROLL}}
\newcommand\ACONTINUE{\AgdaInductiveConstructor{CONTINUE}}
\newcommand\Aback{\AgdaInductiveConstructor{\ensuremath{`}}}
\newcommand\Amanyunaryp{\AgdaFunction{many-unaryp}}
\newcommand\Arestart{\AgdaFunction{restart}}
\newcommand\ASerialize{\AgdaRecord{Serialize}}
\newcommand\ARawMonad{\AgdaRecord{RawMonad}}
\newcommand\AReaderT{\AgdaRecord{ReaderT}}
\newcommand\AStateT{\AgdaRecord{StateT}}
\newcommand\Aput{\AgdaFunction{put}}
\newcommand\Amodify{\AgdaFunction{modify}}
\newcommand\Aget{\AgdaFunction{get}}
\newcommand\Aproject{\AgdaFunction{project}}
\newcommand\AlocateSplit{\AgdaFunction{locate-split}}
\newcommand\Aadjust{\AgdaFunction{adjust}}
\newcommand\Aid{\AgdaFunction{id}}
\newcommand\AprimSend{\AgdaFunction{primSend}}
\newcommand\AprimRecv{\AgdaFunction{primRecv}}
\newcommand\Aleafp{\AgdaFunction{leafp}}
\newcommand\Abranchp{\AgdaFunction{branchp}}
\newcommand\Atreep{\AgdaFunction{treep}}
\newcommand\AIntTree{\AgdaDatatype{IntTree}}
\newcommand\AIntTreeF{\AgdaFunction{IntTreeF}}
\newcommand\ACLeaf{\AgdaInductiveConstructor{Leaf}}
\newcommand\ACBranch{\AgdaInductiveConstructor{Branch}}
\newcommand\AtoN{\AgdaFunction{toℕ}}
\newcommand\Asplit{\AgdaFunction{split}}
\newcommand\AisValue{\AgdaDatatype{isValue}}
\newcommand\AValue{\AgdaDatatype{Value}}
\newcommand\Ajoin{\AgdaFunction{join}}
\newcommand\AExpr{\AgdaFunction{Expr}}
\newcommand\ACExpr{\AgdaFunction{CExpr}}
\newcommand\Acont{\AgdaFunction{cont}}
\newcommand\AlevelEnv{\AgdaFunction{levelEnv}}
\newcommand\Adollar{\AgdaOperator{\AgdaFunction{\AgdaUnderscore{}\$\AgdaUnderscore{}}}}
\newcommand\AVSem{\AgdaOperator{\AgdaFunction{𝓥⟦\AgdaUnderscore{}⟧}}}
\newcommand\AESem{\AgdaOperator{\AgdaFunction{𝓔⟦\AgdaUnderscore{}⟧}}}
\newcommand\AESemx[1]{\AgdaOperator{\AgdaFunction{𝓔⟦#1⟧}}}
\newcommand\AGSem{\AgdaOperator{\AgdaDatatype{𝓖⟦\AgdaUnderscore{}⟧}}}
\newcommand\ATSem{\AgdaOperator{\AgdaDatatype{𝓣⟦\AgdaUnderscore{}⟧}}}
\newcommand\Ainn{\AgdaDatatype{inn}}
\newcommand\ADEnv{\AgdaDatatype{Env\ensuremath{^*}}}
\newcommand\AapplyEnv{\AgdaFunction{apply-env}}
\newcommand\Alookup{\AgdaFunction{lookup}}
\newcommand\AREL{\AgdaFunction{REL}}
\newcommand\ARelEnv{\AgdaFunction{RelEnv}}
\newcommand{\AV}{\AgdaFunction{𝓥}}
\newcommand{\Asubst}{\AgdaFunction{subst}}
\newcommand{\Asubstlo}{\AgdaFunction{substl$\omega$}}
\newcommand{\Acong}{\AgdaFunction{cong}}
\newcommand{\Atrans}{\AgdaFunction{trans}}
\newcommand{\Arefl}{\AgdaInductiveConstructor{refl}}
\newcommand{\AValueDown}{\AgdaFunction{Value-\ensuremath{\Downarrow}}}
\newcommand{\AGLookup}{\AgdaFunction{𝓖-lookup}}
\newcommand{\ACsubClosed}{\AgdaFunction{Csub-closed}}
\newcommand{\Ahere}{\AgdaInductiveConstructor{here}}
\newcommand{\Athere}{\AgdaInductiveConstructor{there}}
\newcommand{\Atskip}{\AgdaInductiveConstructor{tskip}}
\newcommand{\ATwk}{\AgdaFunction{Twk}}
\newcommand{\Anull}{\AgdaInductiveConstructor{∅}}
\newcommand{\ATRen}{\AgdaFunction{TRen}}
\newcommand{\ATSub}{\AgdaFunction{TSub}}
\newcommand{\ATren}{\AgdaFunction{Tren}}
\newcommand{\ATsub}{\AgdaFunction{Tsub}}
\newcommand{\ATliftR}{\AgdaFunction{Tliftᵣ}}
\newcommand{\ATliftS}{\AgdaFunction{Tliftₛ}}
\newcommand{\ATidR}{\AgdaFunction{Tidᵣ}}
\newcommand{\ATidS}{\AgdaFunction{Tidₛ}}
\newcommand{\AERen}{\AgdaFunction{ERen}}
\newcommand{\AESub}{\AgdaFunction{ESub}}
\newcommand{\AEren}{\AgdaFunction{Eren}}
\newcommand{\AEsub}{\AgdaFunction{Esub}}
\newcommand{\AEliftR}{\AgdaFunction{Eliftᵣ}}
\newcommand{\AEliftS}{\AgdaFunction{Eliftₛ}}
\newcommand{\AEliftRL}{\AgdaFunction{Eliftᵣ-l}}
\newcommand{\AEliftSL}{\AgdaFunction{Eliftₛ-l}}
\newcommand{\AEidR}{\AgdaFunction{Eidᵣ}}
\newcommand{\AEidS}{\AgdaFunction{Eidₛ}}
\newcommand{\AFusionTSubTSub}{\AgdaFunction{fusion-TSub-TSub}}
\newcommand{\AFusionESubESub}{\AgdaFunction{fusion-ESub-ESub}}
\newcommand{\ALRVsub}{\AgdaFunction{LRVsub}}

\newcommand*\ACode[1]{\AgdaFontStyle{#1}}
\newcommand*\AField[1]{\AgdaField{#1}}
\newcommand*\ACon[1]{\AgdaInductiveConstructor{#1}}
\newcommand*\AKw[1]{\AgdaKeyword{#1}}
\newcommand*\AFun[1]{\AgdaFunction{#1}}


%%% Local Variables:
%%% mode: latex
%%% TeX-master: "main-icfp24"
%%% End:


\usetheme{Madrid}

\title{Relating System F Semantics in Agda}
\author[Saffrich, Thiemann, Weidner] {
  Hannes Saffrich \and 
  Peter Thiemann \and
  Marius Weidner
}
\institute{University of Freiburg}
\date{April 28, 2024 (40. GI Workshop, Bad Honnef)}

\newcommand{\SubItem}[1]{
    {\setlength\itemindent{15pt} \item[-] #1}
}

\AtBeginSection[]{%
  \begin{frame}<beamer>
    \frametitle{Outline}
    \tableofcontents[currentsection]%[sectionstyle=show/show,subsectionstyle=hide/show/hide]
  \end{frame}
  \addtocounter{framenumber}{-1}% If you don't want them to affect the slide number
}

\newenvironment{AgdaBlock}{
  \vspace{\AgdaEmptySkip}
  \AgdaNoSpaceAroundCode{}
}{
  \AgdaSpaceAroundCode{}
}

\begin{document}
\begin{frame}{\null}
  \titlepage 
\end{frame}

\begin{frame}[fragile]
  \frametitle{Overview}
  \begin{itemize}
    \item We consider Leivant's finitely stratified System F $SF_2$ \cite{DBLP:journals/iandc/Leivant91}
    \item We define an intrinsically typed syntax for $SF_2$
    \item We define small-step, big-step \& denotational semantics for $SF_2$
    \item .. and relate those three semantics with each other
    \item .. in Agda
  \end{itemize}
\end{frame}

\begin{frame}
  \frametitle{Intrinsically Typed Syntax for $SF_2$}
  \framesubtitle{Types}
  \TFLEnv
  \TFType
  \begin{itemize}
    \item Polymorphic lambda calculus \cite{girard72:_inter,DBLP:conf/programm/Reynolds74}
    \item Each type has a level
    \item Quantification only possible over types at lower level
    \item Predicativity is retained
    \item .. to allow for set theoretic semantics
  \end{itemize}
\end{frame}

\begin{frame}
  \frametitle{Intrinsically Typed Syntax for $SF_2$}
  \framesubtitle{Type Contexts}
  \TFTVEnv
  \TFinn
  \begin{itemize}
    \item Single environment for type and expression variables inspired by \cite{DBLP:conf/mpc/ChapmanKNW19}
  \end{itemize}
\end{frame}

\begin{frame}
  \frametitle{Intrinsically Typed Syntax for $SF_2$}
  \framesubtitle{Expressions}
  \TFExprExcerpt
\end{frame}

\begin{frame}
  \frametitle{Operational Semantics of $SF_2$}
  \framesubtitle{Small \& Big Step Semantics}
  \SmallStepSingleReductionExcerpt
  \SmallStepSemanticsExcerpt
\end{frame}

\begin{frame}
  \frametitle{Denotational Semantics of $SF_2$}
  \framesubtitle{Types}
  \TFTEnv
  \TFTSem
  \begin{itemize}
    \item Leivant’s levels correspond to Agda’s universe levels
    \item .. and thus \AgdaFunction{Env*} need to live in \AgdaFunction{Setω}!
  \end{itemize}
\end{frame}

\begin{frame}
  \frametitle{Problem \#1: $Set \omega$ Equality}
  % \begin{itemize}
  %   \item \AgdaFunction{Setω} is Agda’s sort that contains \AgdaFunction{Set} $\ell$ for all $\ell :$ \AgdaFunction{Level}
  %   \item Some proofs argue about equality in types of sort \AgdaFunction{Setω}:
  %     \TFSingleSubstPreserves
  %   \item 
  % \end{itemize}
  \begin{exampleblock}{Proposal}
    Extend the level hierarchy in Agda from $\omega$ to $\epsilon_0$.
  \end{exampleblock}
\end{frame}

\begin{frame}
  \frametitle{Denotational Semantics of $SF_2$}
  \framesubtitle{Expressions}
  \TFVEnv
  \TFExprSem
\end{frame}

\begin{frame}[fragile]
  \frametitle{Relating Operational and Denotational Semantics}
  \framesubtitle{Birds eye view on the theorems}
  \begin{tikzcd}
    small step \arrow[rr, "simulates", dashed] \arrow[rdd, bend left] &                                                                  & big step \arrow[ldd, "soundness \ "', bend right] \\
                                                                      &                                                                  &                                                   \\
                                                                      & denotational \arrow[ruu, "adequacy \ (using \ LR)"', bend right] &                                                  
    \end{tikzcd}
    \vspace{5mm}
    \begin{itemize}
      \item Simulation is easy: the reflexive transitive closure of the small-step relation corresponds to big-step semantics
      \item Soundness can be proven by induction
      \item .. but adequacy requires a binary logical relation 
    \end{itemize}
\end{frame}

\begin{frame}[fragile]
  \frametitle{Soundness}
  \BigStepSoundnessType
  \SmallStepSoundness
  \vspace{-12.5mm} 
  \SmallStepSoundnessProofExcerpt
\end{frame}

\begin{frame}[fragile]
  \frametitle{Problem \#2: Subst Hell}
  \begin{exampleblock}{Proposal}
    Build a solver for equality reasoning over \AgdaFunction{subst} terms. 
  \end{exampleblock}
\end{frame}

\begin{frame}[fragile]
  \frametitle{Adequacy}
  \FundamentalAdequacyType
  \vspace{-12.5mm} 
  \FundamentalAdequacyBody
  \begin{itemize}
    \item The \AgdaFunction{adequacy} theorem directly follows from the \AgdaFunction{fundamental} theorem that itself is mostly based on \cite{DBLP:journals/corr/abs-1907-11133,ahmed23:_oplss}
    \item .. and also suffers from `subst hell'
  \end{itemize}
\end{frame}


\begin{frame}[fragile]
  \frametitle{Conclusion}
  \framesubtitle{}
  \begin{itemize}
    \item We mechanized the following theorems in Agda: 
  \end{itemize}
  \begin{tikzcd}
    small step \arrow[rr, "simulates", dashed] \arrow[rdd, bend left] &                                                                  & big step \arrow[ldd, "soundness \ "', bend right] \\
                                                                      &                                                                  &                                                   \\
                                                                      & denotational \arrow[ruu, "adequacy \ (using \ LR)"', bend right] &                                                  
  \end{tikzcd}
  \begin{itemize} 
    \item On paper those theorems are well studied
    \item .. but mechanizing them opens up technical challenges:
    \begin{itemize} 
      \item Soundness and adequacy (i.e. the logical relation) theorems require fusion lemmas for type substitution indexed expression substitutions, leading to `subst hell'
      \item Denotational semantics for languages with level quantification make use of \AgdaFunction{Setω} and require reasoning about hetero- \& homogeneous `$\omega$ equality'
    \end{itemize}
  \end{itemize}
  \begin{verbatim} 
  >> find . -name '*.agda' -print0 | xargs -0 cat | wc -l
  10889
  \end{verbatim}
\end{frame}

\begin{frame}[fragile]
  \frametitle{Sources}  
  \bibliographystyle{ACM-Reference-Format} 
  \bibliography{references}
\end{frame}

\end{document}
