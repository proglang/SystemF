\documentclass[dvipsnames,aspectratio=169,pdftex]{beamer}
\usepackage{agda}
\usepackage{stmaryrd}
\usepackage{xcolor}
\usepackage{txfonts}
\usepackage{tikz}
\usetikzlibrary{cd}
\usepackage{agda-generated}
\input{unicodeletters}
\newcommand\Aamp{\AgdaFunction{\ensuremath{\&}}}
\newcommand\Att{\AgdaInductiveConstructor{tt}}
\newcommand\Ado{\AgdaKeyword{do}}
\newcommand\AZ{\AgdaDatatype{ℤ}}
\newcommand\Asuc{\AgdaInductiveConstructor{suc}}
\newcommand\Azero{\AgdaInductiveConstructor{zero}}
\newcommand\Anat{\AgdaInductiveConstructor{nat}}
\newcommand\Aint{\AgdaInductiveConstructor{int}}
\newcommand\Abool{\AgdaInductiveConstructor{bool}}
\newcommand\Atend{\AgdaInductiveConstructor{end}}
\newcommand\Atsend[2]{\AgdaInductiveConstructor{‼{\textcolor{black}{\ensuremath{#1}}}∙{\textcolor{black}{\ensuremath{#2}}}}}
\newcommand\Atrecv[2]{\AgdaInductiveConstructor{⁇{\textcolor{black}{\ensuremath{#1}}}∙{\textcolor{black}{\ensuremath{#2}}}}}
\newcommand\Atcfsend[1]{\AgdaInductiveConstructor{‼{\textcolor{black}{\ensuremath{#1}}}}}
\newcommand\Atcfrecv[1]{\AgdaInductiveConstructor{⁇{\textcolor{black}{\ensuremath{#1}}}}}
\newcommand\Atcfcomp[2]{\AgdaInductiveConstructor{{\textcolor{black}{\ensuremath{#1}}}⨟{\textcolor{black}{\ensuremath{#2}}}}}
\newcommand\Atcfskip{\AgdaInductiveConstructor{skip}}
\newcommand\ACSKIP{\AgdaInductiveConstructor{SKIP}}
\newcommand\ACEND{\AgdaInductiveConstructor{END}}
\newcommand\ACCLOSE{\AgdaInductiveConstructor{CLOSE}}
\newcommand\ACfork{\AgdaInductiveConstructor{fork}}
\newcommand\ACconnect{\AgdaInductiveConstructor{connect}}
\newcommand\ACterminate{\AgdaInductiveConstructor{terminate}}
\newcommand\ACdelegateIN{\AgdaInductiveConstructor{delegateIN}}
\newcommand\ACdelegateOUT{\AgdaInductiveConstructor{delegateOUT}}
\newcommand\ACtransmit{\AgdaInductiveConstructor{transmit}}
\newcommand\ACbranch{\AgdaInductiveConstructor{branch}}
\newcommand\ACclose{\AgdaInductiveConstructor{close}}
\newcommand\ACSEND{\AgdaInductiveConstructor{SEND}}
\newcommand\ACRECV{\AgdaInductiveConstructor{RECV}}
\newcommand\ACSELECT{\AgdaInductiveConstructor{SELECT}}
\newcommand\ACCHOICE{\AgdaInductiveConstructor{CHOICE}}
\newcommand\AFin{\AgdaDatatype{Fin}}
\newcommand\AXCommand{\AgdaDatatype{XCmd}}
\newcommand\ACommand{\AgdaDatatype{Cmd}}
\newcommand\ACommandStack{\AgdaDatatype{CmdStack}}
\newcommand\ASession{\AgdaDatatype{Session}}
\newcommand\ASplit{\AgdaDatatype{Split}}
\newcommand\AMSession{\AgdaDatatype{MSession}}
\newcommand\ASet{\AgdaDatatype{Set}}
\newcommand\ASetOne{\AgdaDatatype{Set$_1$}}
\newcommand\ASeto{\AgdaDatatype{Setω}}
\newcommand\Abinaryp{\AgdaFunction{binaryp}}
\newcommand\Aunaryp{\AgdaFunction{unaryp}}
\newcommand\ACheck{\AgdaFunction{Check}}
\newcommand\ACausality{\AgdaFunction{Causality}}
\newcommand\ACheckDual{\AgdaFunction{CheckDual0}}
\newcommand\Aexecutor{\AgdaFunction{exec}}
\newcommand\Aexec{\AgdaFunction{exec}}
\newcommand\AIO{\AgdaFunction{IO}}
\newcommand\Amu{\AgdaInductiveConstructor{\ensuremath{\mu}}}
\newcommand\AMU{\AgdaInductiveConstructor{LOOP}}
\newcommand\AUNROLL{\AgdaInductiveConstructor{UNROLL}}
\newcommand\ACONTINUE{\AgdaInductiveConstructor{CONTINUE}}
\newcommand\Aback{\AgdaInductiveConstructor{\ensuremath{`}}}
\newcommand\Amanyunaryp{\AgdaFunction{many-unaryp}}
\newcommand\Arestart{\AgdaFunction{restart}}
\newcommand\ASerialize{\AgdaRecord{Serialize}}
\newcommand\ARawMonad{\AgdaRecord{RawMonad}}
\newcommand\AReaderT{\AgdaRecord{ReaderT}}
\newcommand\AStateT{\AgdaRecord{StateT}}
\newcommand\Aput{\AgdaFunction{put}}
\newcommand\Amodify{\AgdaFunction{modify}}
\newcommand\Aget{\AgdaFunction{get}}
\newcommand\Aproject{\AgdaFunction{project}}
\newcommand\AlocateSplit{\AgdaFunction{locate-split}}
\newcommand\Aadjust{\AgdaFunction{adjust}}
\newcommand\Aid{\AgdaFunction{id}}
\newcommand\AprimSend{\AgdaFunction{primSend}}
\newcommand\AprimRecv{\AgdaFunction{primRecv}}
\newcommand\Aleafp{\AgdaFunction{leafp}}
\newcommand\Abranchp{\AgdaFunction{branchp}}
\newcommand\Atreep{\AgdaFunction{treep}}
\newcommand\AIntTree{\AgdaDatatype{IntTree}}
\newcommand\AIntTreeF{\AgdaFunction{IntTreeF}}
\newcommand\ACLeaf{\AgdaInductiveConstructor{Leaf}}
\newcommand\ACBranch{\AgdaInductiveConstructor{Branch}}
\newcommand\AtoN{\AgdaFunction{toℕ}}
\newcommand\Asplit{\AgdaFunction{split}}
\newcommand\AisValue{\AgdaDatatype{isValue}}
\newcommand\AValue{\AgdaDatatype{Value}}
\newcommand\Ajoin{\AgdaFunction{join}}
\newcommand\AExpr{\AgdaFunction{Expr}}
\newcommand\ACExpr{\AgdaFunction{CExpr}}
\newcommand\Acont{\AgdaFunction{cont}}
\newcommand\AlevelEnv{\AgdaFunction{levelEnv}}
\newcommand\Adollar{\AgdaOperator{\AgdaFunction{\AgdaUnderscore{}\$\AgdaUnderscore{}}}}
\newcommand\AVSem{\AgdaOperator{\AgdaFunction{𝓥⟦\AgdaUnderscore{}⟧}}}
\newcommand\AESem{\AgdaOperator{\AgdaFunction{𝓔⟦\AgdaUnderscore{}⟧}}}
\newcommand\AESemx[1]{\AgdaOperator{\AgdaFunction{𝓔⟦#1⟧}}}
\newcommand\AGSem{\AgdaOperator{\AgdaDatatype{𝓖⟦\AgdaUnderscore{}⟧}}}
\newcommand\ATSem{\AgdaOperator{\AgdaDatatype{𝓣⟦\AgdaUnderscore{}⟧}}}
\newcommand\Ainn{\AgdaDatatype{inn}}
\newcommand\ADEnv{\AgdaDatatype{Env\ensuremath{^*}}}
\newcommand\AapplyEnv{\AgdaFunction{apply-env}}
\newcommand\Alookup{\AgdaFunction{lookup}}
\newcommand\AREL{\AgdaFunction{REL}}
\newcommand\ARelEnv{\AgdaFunction{RelEnv}}
\newcommand{\AV}{\AgdaFunction{𝓥}}
\newcommand{\Asubst}{\AgdaFunction{subst}}
\newcommand{\Asubstlo}{\AgdaFunction{substl$\omega$}}
\newcommand{\Acong}{\AgdaFunction{cong}}
\newcommand{\Atrans}{\AgdaFunction{trans}}
\newcommand{\Arefl}{\AgdaInductiveConstructor{refl}}
\newcommand{\AValueDown}{\AgdaFunction{Value-\ensuremath{\Downarrow}}}
\newcommand{\AGLookup}{\AgdaFunction{𝓖-lookup}}
\newcommand{\ACsubClosed}{\AgdaFunction{Csub-closed}}
\newcommand{\Ahere}{\AgdaInductiveConstructor{here}}
\newcommand{\Athere}{\AgdaInductiveConstructor{there}}
\newcommand{\Atskip}{\AgdaInductiveConstructor{tskip}}
\newcommand{\ATwk}{\AgdaFunction{Twk}}
\newcommand{\Anull}{\AgdaInductiveConstructor{∅}}
\newcommand{\ATRen}{\AgdaFunction{TRen}}
\newcommand{\ATSub}{\AgdaFunction{TSub}}
\newcommand{\ATren}{\AgdaFunction{Tren}}
\newcommand{\ATsub}{\AgdaFunction{Tsub}}
\newcommand{\ATliftR}{\AgdaFunction{Tliftᵣ}}
\newcommand{\ATliftS}{\AgdaFunction{Tliftₛ}}
\newcommand{\ATidR}{\AgdaFunction{Tidᵣ}}
\newcommand{\ATidS}{\AgdaFunction{Tidₛ}}
\newcommand{\AERen}{\AgdaFunction{ERen}}
\newcommand{\AESub}{\AgdaFunction{ESub}}
\newcommand{\AEren}{\AgdaFunction{Eren}}
\newcommand{\AEsub}{\AgdaFunction{Esub}}
\newcommand{\AEliftR}{\AgdaFunction{Eliftᵣ}}
\newcommand{\AEliftS}{\AgdaFunction{Eliftₛ}}
\newcommand{\AEliftRL}{\AgdaFunction{Eliftᵣ-l}}
\newcommand{\AEliftSL}{\AgdaFunction{Eliftₛ-l}}
\newcommand{\AEidR}{\AgdaFunction{Eidᵣ}}
\newcommand{\AEidS}{\AgdaFunction{Eidₛ}}
\newcommand{\AFusionTSubTSub}{\AgdaFunction{fusion-TSub-TSub}}
\newcommand{\AFusionESubESub}{\AgdaFunction{fusion-ESub-ESub}}
\newcommand{\ALRVsub}{\AgdaFunction{LRVsub}}

\newcommand*\ACode[1]{\AgdaFontStyle{#1}}
\newcommand*\AField[1]{\AgdaField{#1}}
\newcommand*\ACon[1]{\AgdaInductiveConstructor{#1}}
\newcommand*\AKw[1]{\AgdaKeyword{#1}}
\newcommand*\AFun[1]{\AgdaFunction{#1}}


%%% Local Variables:
%%% mode: latex
%%% TeX-master: "main-icfp24"
%%% End:


\usetheme{Madrid}

\title{Relating System F Semantics in Agda}
\author[Saffrich, Thiemann, Weidner] {
  Hannes Saffrich \and 
  Peter Thiemann \and
  Marius Weidner
}
\institute{University of Freiburg}
\date{April 28, 2024 (40. GI Workshop, Bad Honnef)}

\AtBeginSection[]{%
  \begin{frame}<beamer>
    \frametitle{Outline}
    \tableofcontents[currentsection]%[sectionstyle=show/show,subsectionstyle=hide/show/hide]
  \end{frame}
  \addtocounter{framenumber}{-1}% If you don't want them to affect the slide number
}

\newenvironment{AgdaBlock}{
  \vspace{\AgdaEmptySkip}
  \AgdaNoSpaceAroundCode{}
}{
  \AgdaSpaceAroundCode{}
}

\begin{document}
\begin{frame}{\null}
  \titlepage 
\end{frame}

\begin{frame}[fragile]
  \frametitle{Overview}
  \begin{itemize}
    \item We consider Leivant's [cite] finitely stratified System F $SF_2$.
    \item We define an intrinsically typed syntax for $SF_2$.
    \item We define big-step, small-step \& denotational semantics for $SF_2$
    \item .. and relate those three semantics with each other
    \item .. in Agda
  \end{itemize}
\end{frame}

\begin{frame}[fragile]
  \frametitle{Relating Semantics}
  \framesubtitle{Birds eye view on the theorems}
  \begin{tikzcd}
    small step \arrow[rr, "simulates"] \arrow[rdd, bend left] &                                                                  & big step \arrow[ldd, "soundness \ "', bend right] \\
                                                              &                                                                  &                                                   \\
                                                              & denotational \arrow[ruu, "adequacy \ (using \ LR)"', bend right] &                                                  
  \end{tikzcd}
  \vspace{5mm}
  \begin{itemize} 
    \item Soundness requires fusion of expression \& type substitutions
    \item Adequacy additionally requires logical relation
  \end{itemize}
\end{frame}

\begin{frame}
  \frametitle{Intrinsic $SF_2$}
  \framesubtitle{Types}
  \TFType
  \begin{itemize}
    \item Polymorphic lambda calculus [cite]
    \item Each type has a level
    \item Quantification only possible over types at lower level
    \item Predicativity is retained
    \item .. to allow for set theoretic semantics
  \end{itemize}
\end{frame}

\begin{frame}
  \frametitle{Intrinsic $SF_2$}
  \framesubtitle{Type Contexts}
  \TFTVEnv
  \TFinn
  \begin{itemize}
    \item Single environment for type and expression variables [cite]
  \end{itemize}
\end{frame}

\begin{frame}
  \frametitle{Intrinsic $SF_2$}
  \framesubtitle{Expressions}
  \TFExprExcerpt
\end{frame}

\begin{frame}
  \frametitle{Semantics of $SF_2$}
  \framesubtitle{Operational}
  \SmallStepSingleReductionExcerpt
  \SmallStepSemanticsExcerpt
\end{frame}

% denotational 
% soundness
% adequacy + log rel (cite ahmed et al)
% subst hell
% sources


% - STLC, denotational, big, small
% - -> system f
% - relating different semantics
% - logical relation (?)
% - difficulties with indexed substitutions -> homo vs hetero

% \begin{frame}
%   \frametitle{Types of System F}
%   \framesubtitle{Intrinsically scoped encoding}
%   \SFType
%   \begin{itemize}
%   \item Polymorphic lambda calculus (Girard, Reynolds)
%   \item Impredicative: quantification extends over all types
%   \item In $T = \forall \alpha.S$, $\alpha$ ranges over all types including $T$
%   \item Precludes a set-theoretic semantics
%   \end{itemize}
% \end{frame}
% \begin{frame}
%   \begin{center}
%     \includegraphics[scale=0.25]{images/FinitelyStratifiedPolymorphism.png}
%   \end{center}
% \end{frame}
% \begin{frame}
%   \frametitle{Finite Stratification}
%   \framesubtitle{Intrinsically leveled encoding}
%   \begin{itemize}
%   \item Each type has a level (i.e., a natural number)
%   \item Quantification is only possible over types at lower level
%   \item Predicativity is retained
%   \item Simple set-theoretic semantics
%   \end{itemize}
%   \pause
%   \TFType
%   \pause\vspace{-2\baselineskip}
%   \TFlevel
% \end{frame}
% \begin{frame}
%   \frametitle{Semantics of Types}
%   \begin{itemize}
%   \item Leivant's levels correspond to Agda's universe levels, so \dots
%   \end{itemize}
%   \pause
%   \TFTEnvP
%   \pause\vspace{-2\baselineskip}
%   \TFTSemP
%   \begin{itemize}
%   \item Works because we're using \AgdaFunction{Level} in the syntax!
%   \end{itemize}
% \end{frame}
% \begin{frame}
%   \frametitle{Type Environments and Variables}
%   \begin{itemize}
%   \item A single environment for type and term variables
%   \item Encoding inspired by \emph{System F in Agda, for fun and profit} (Chapman et al, MPC 2019)
%   \end{itemize}
%   \TFTVEnv
%   \pause\vspace{-2\baselineskip}
%   \TFCleanerinn
% \end{frame}
% \begin{frame}
%   \frametitle{Syntax of Expressions (Excerpt)}
%   \framesubtitle{Intrinsically typed encoding}
%   \TFCleanExpr
% \end{frame}
% \begin{frame}
%   \frametitle{Set-Theoretic Semantics of Expressions}
%   \TFExprSem
%   where
%   \TFVEnv
% \end{frame}
% \begin{frame}
%   \frametitle{Set woes}
%   \begin{itemize}
%   \item \AgdaFunction{Setω} is Agda's sort that contains \AgdaFunction{Set ℓ}, for all \AgdaFunction{ℓ}.
%   \item Some proofs argue about equality in types of sort \AgdaFunction{Setω}:
%     \TFSingleSubstPreserves
%   \item This equality is easy to define, but leads to a proliferation of uninteresting copies of library functions like \AgdaFunction{cong}, \AgdaFunction{subst}, \dots; equational reasoning; extensionality axioms, and so on.
%   \end{itemize}
% \end{frame}
% \begin{frame}
%   \frametitle{Set woes II}
%   \begin{itemize}
%   \item Consider extending the calculus with level-polymorphism.
%   \item Can be modeled in Agda, but forces a departure from the simple semantics of types.
%   \item A level-polymorphic function is a member of \AgdaFunction{Setω}, but that means the semantics of a type can no longer be in  \AgdaFunction{Setω}!
%   \item It must be in \AgdaFunction{Setω₁} $\Longrightarrow$ can't index types by levels, need more equalities, and so on.
%   \end{itemize}
%   \begin{exampleblock}{Wish to the  Agda maintainers}
%     \begin{itemize}
%     \item Extend \AgdaFunction{Level} to include a larger subset of ordinals.
%     \item Leivant's 1989 paper \emph{Stratified Polymorphism} suggests
%       a useful subset.
%     \item Recent work by Bezem, Coquand, Dybjer, Escardo (TYPES 2022)
%     \end{itemize}
%   \end{exampleblock}
% \end{frame}
% \begin{frame}
%   \frametitle{Where do we go from here?}
%   \begin{itemize}
%   \item All results for Stratified System F apply directly to subsystems like ML.
%   \item Intrinsically typed small-step and big-step semantics.
%   \item Soundness with respect to denotational semantics (stratification required).
%   \item Logical relation and adequacy (stratification required).
%   \end{itemize}
% \end{frame}
\end{document}
